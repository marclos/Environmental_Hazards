\documentclass{tufte-handout}

%\geometry{showframe}% for debugging purposes -- displays the margins
\usepackage{verbatim}
\usepackage{amsmath}
\usepackage{natbib}
\bibfont{\small} % Doesn't see to work...

% Set up the images/graphics package
\usepackage{graphicx}
\setkeys{Gin}{width=\linewidth,totalheight=\textheight,keepaspectratio}
% \graphicspath{{graphics/}}

\title{One-way ANOVA: From Calculations to Diagnostics %\thanks{}
}
\author[Marc Los Huertos]{Marc Los Huertos}
\date{}  % if the \date{} command is left out, the current date will be used


% \SweaveOpts{prefix.string=graphics/plot} % Created a "graphics" subdirectory to 

\setsidenotefont{\color{blue}}
% \setcaptionfont{hfont commandsi}
% \setmarginnotefont{\color{blue}}
% \setcitationfont{\color{gray}}

% The following package makes prettier tables.  We're all about the bling!
\usepackage{booktabs}

% Small sections of multiple columns
\usepackage{multicol}

% These commands are used to pretty-print LaTeX commands
% command name -- adds backslash automatically
\newcommand{\docpkg}[1]{\texttt{#1}}% package name
\newcommand{\doccls}[1]{\texttt{#1}}% document class name
\newcommand{\docclsopt}[1]{\texttt{#1}}% document class option name

\begin{document}

\maketitle% this prints the handout title, author, and date
\begin{abstract}
\noindent Analysis of Variance has been the dominant method to analyze treatment effects for nearly 70 years. In this handout, you will learn to do an analysis of variance by hand, then use R to do it with as a calculator, and finally using R to do the whole thing automatically. In this handout, we focus on how we partition the deviation between signal and noise and model valuation as a key step to ensure assumptions for the ANOVA models are met: Independent, Identically Distributed (or homogeneity of variance), and errors are normally distributed. We will also explore the different between planned-comparisons and post-hoc tests. Finally, we will try to understand the "expectations" in the ANOVA framework.  
\end{abstract}

%\printclassoptions

% Setting up the margins, etc for R

































































































































































































