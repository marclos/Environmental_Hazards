%\documentclass{tufte-handout}
\documentclass{article}
\usepackage{hyperref}

\author{Marc Los Huertos}
\title{Does racism cause environmental hazards?}
\usepackage{Sweave}
\begin{document}
\Sconcordance{concordance:Template.tex:Template.Rnw:%
1 6 1 1 0 55 1}

\maketitle

\section{Introduction}

\section{Driving Question}

Does racism cause environmental hazard?

\section{Public Product}

Each student will create a map the evaluates how environmental hazards are spatially distributed and evaluate factors that cause this exposure risks.


\section{Learning Outcomes}

Students will be able to: 

\begin{itemize}
  \item Propose methods chemical concentrations from soil;
  \item Map environmental hazards;
  \item Analyze spatial data using demographics and hazards.

\end{itemize}

\section{Stages of Project}

\section{Refine Driving Question}

\section{Identify Resources Needed}

\subsection{Toxic Release Inventory Program}

The US EPA runs a program called the \href{https://www.epa.gov/toxics-release-inventory-tri-program}{Toxic Release Inventory}. The program relies on emitters of toxic pollutants to report their releases on an annual?? basis. 

\subsection{TRI Chemicals}

The TRI Program requires emitters to report on the following types of chemicals:

\begin{itemize}
  \item Cancer or other chronic human health effects;
  \item Significant adverse acute human health effects; and
  \item Significant adverse environmental effects.
\end{itemize}

The \href{https://www.epa.gov/sites/production/files/2015-11/tri_chemical_list_for_ry15_11_5_2015_1.xlsx}{current TRI toxic chemical list} contains 594 individually-listed chemicals and 31 chemical categories (including four categories containing 68 specifically-listed chemicals). 

\section{Proposed Sampling and Analysis Methods}


\section{Create Public Product}



\end{document}
