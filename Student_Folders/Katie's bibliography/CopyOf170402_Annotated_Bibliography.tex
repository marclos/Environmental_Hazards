\documentclass{article}\usepackage[]{graphicx}\usepackage[]{color}
%% maxwidth is the original width if it is less than linewidth
%% otherwise use linewidth (to make sure the graphics do not exceed the margin)
\makeatletter
\def\maxwidth{ %
  \ifdim\Gin@nat@width>\linewidth
    \linewidth
  \else
    \Gin@nat@width
  \fi
}
\makeatother

\definecolor{fgcolor}{rgb}{0.345, 0.345, 0.345}
\newcommand{\hlnum}[1]{\textcolor[rgb]{0.686,0.059,0.569}{#1}}%
\newcommand{\hlstr}[1]{\textcolor[rgb]{0.192,0.494,0.8}{#1}}%
\newcommand{\hlcom}[1]{\textcolor[rgb]{0.678,0.584,0.686}{\textit{#1}}}%
\newcommand{\hlopt}[1]{\textcolor[rgb]{0,0,0}{#1}}%
\newcommand{\hlstd}[1]{\textcolor[rgb]{0.345,0.345,0.345}{#1}}%
\newcommand{\hlkwa}[1]{\textcolor[rgb]{0.161,0.373,0.58}{\textbf{#1}}}%
\newcommand{\hlkwb}[1]{\textcolor[rgb]{0.69,0.353,0.396}{#1}}%
\newcommand{\hlkwc}[1]{\textcolor[rgb]{0.333,0.667,0.333}{#1}}%
\newcommand{\hlkwd}[1]{\textcolor[rgb]{0.737,0.353,0.396}{\textbf{#1}}}%
\let\hlipl\hlkwb

\usepackage{framed}
\makeatletter
\newenvironment{kframe}{%
 \def\at@end@of@kframe{}%
 \ifinner\ifhmode%
  \def\at@end@of@kframe{\end{minipage}}%
  \begin{minipage}{\columnwidth}%
 \fi\fi%
 \def\FrameCommand##1{\hskip\@totalleftmargin \hskip-\fboxsep
 \colorbox{shadecolor}{##1}\hskip-\fboxsep
     % There is no \\@totalrightmargin, so:
     \hskip-\linewidth \hskip-\@totalleftmargin \hskip\columnwidth}%
 \MakeFramed {\advance\hsize-\width
   \@totalleftmargin\z@ \linewidth\hsize
   \@setminipage}}%
 {\par\unskip\endMakeFramed%
 \at@end@of@kframe}
\makeatother

\definecolor{shadecolor}{rgb}{.97, .97, .97}
\definecolor{messagecolor}{rgb}{0, 0, 0}
\definecolor{warningcolor}{rgb}{1, 0, 1}
\definecolor{errorcolor}{rgb}{1, 0, 0}
\newenvironment{knitrout}{}{} % an empty environment to be redefined in TeX

\usepackage{alltt}
\usepackage{natbib}
\usepackage{hyperref}
\usepackage{bibentry}

\bibliographystyle{plainnat}
\setcitestyle{open={(},close={)}}

\author{EA30 -- Spring 2017}
\title{Annotated Bibliography}
\IfFileExists{upquote.sty}{\usepackage{upquote}}{}
\begin{document}
\maketitle
\nobibliography{Pb_literature}

\section{Abstract}



\tableofcontents

\section{Instructions}

Please complete the annotated bibliography by the 17th of April.

\subsection{Searching for Academic Sources}

There are numerous electronic sources to evaluate and obtain academic sources of information. Usually, we consider peer reviewed articles to have the highest quality scholarship. As a rule of thumb, this is a good start, but there is also a great deal of variation between sources, even journal titles can vary in quality. Thus, it's best to evaluate a range of sources and appreciate the subtle differences in quality and prestige.

First, you should search CUC databases using key words, such as Pb and lead. I admit the lead is tricky, because it's not a useful term due it a double meaning. I suggest heavy metals or trace metals with the other key words, i.e. fate and transport, toxicity, etc, depending on the topic you selected.

Below are links to some useful databases:

\begin{itemize}
  \item \href{http://apps.webofknowledge.com.ccl.idm.oclc.org/}{Web of Science}
  \item \href{http://www.jstor.org.ccl.idm.oclc.org/}{JStore}
  \item \href{https://scholar-google-com.ccl.idm.oclc.org/}{Google Scholar}
\end{itemize}


\subsection{Writing an Annotated Bibliography}

After adding your citation to the Pb\_literature.bib, using the BibTeX format, cite your reference using the syntax below. Then, write a concise annotation that summarizes the central theme and scope of the book or article. Include one or more sentences that (a) evaluate the authority or background of the author, (b) comment on the intended audience, (c) compare or contrast this work with another you have cited, or (d) explain how this work illuminates your bibliography topic.

\subsection{Implementing in \LaTeX}

\LaTeX is a software program used for desktop publishing. With an eye for detail, the program was developed to give the author a great deal of control. In contrast, Microsoft Word is designed to have lots of options, but these seem to get in the way of controlling the outcome. 

\LaTeX is an open source program and relies on specific formatting commands that begin with a backslash. For example to start a new section heading, we use \verb!\section{section name}! and \verb!\subsection{subsection name}! to create a subsection heading.

Below might be an example of what we are trying to accomplish using our Rstudio resources. The \LaTeX command for the citation is \verb!\bibentry{key}!, where key is identifies the citation based on the BibTeX citation entry as shown below.

However, we also need to add the citation to the *.bib file, in this case the file is called ``Pb\_literature.bib''. Open the file. You can see each entry within the curly brackets. Also, it starts with the definition of the entry, e.g. article, book, conference proceedings. We can easily add the citation using the databases cited above. All you need to do is search for the citation manager or citation tool and select a citation format ``bibtex''. Copy and paste the bibtex format (try to put it in the correct location, i.e. alphabetic order). Note the ``key'' is the first word after the citation definition. 

\subsection{Annotated Example}

\bigskip
%\noindent \bibentry{barltrop1975absorption} . Summarizes important infomormation about Pb. This paper reviews work that span x years starting with early \ldots and noting recent information that includes \ldots. \cite{barltrop1975absorption} first describe xyz and how these issues have been largely \ldots.

\bigskip


To help the readers, you might create subsection and even subsubsections, using \verb!\subsection{subsection name}! and \verb!\subsubsection{subsubsection name}!. 


\section{Industrial Sources (Khalil)}


\section{Use in Industry (except gasoline) (Caudia)}

\section{``Ethyl'' gas and airplane fuels (Viraj)}






\section{Aquatic transport (Olivia)}

\section{Sinks (Katie)}

\noindent \bibentry{agency1990toxicological}.
\medskip

This toxicological profile created by the Agency for Toxic Substances \& Disease Registry and the Environmental Protection Agency details the human health effects of lead exposure. It also provides information on the behavior of lead in soil and sediment and the factors that influence what areas become sinks. For example, temperature, pH, and the presence of humic materials are highly important in determining the mobility of lead in soil. The intended audience for this document are public health professionals, interested private organizations, and the general public. 

\medskip

\noindent \bibentry{bazzano2016long}.

\medskip

This article, featured in the peer-reviewed scientific journal, Atmospheric Environment, discusses the long-range transport of lead particulates to the Norwegian Arctic. The intended audience for this journal are scientists whose work is related to atmospheric science. The article also notes that there is a seasonal trend in lead concentrations. For instance, lead concentrations are higher in the spring and the summer. The lead accumulating in the Norwegian Arctic during the summer is a result of North American industrial emissions. 
\medskip

\noindent \bibentry{amodio2014atmospheric}.

\medskip

This article is featured in Advances in Meteorology, a peer-reviewed, Open Access journal whose primary audience are scientists in the field of meteorology. The article details the mechanisms of atmospheric deposition of heavy metals, such as lead. In addition, the location of deposition is strongly dependent on the climate and the amount of rainfall the area receives. 

\medskip

\noindent \bibentry{coulibaly2015seasonal}.

\medskip

This study, supported by the Japanese Ministry of the Environment, focuses on the effects of long-range transport of lead and other pollutants on Dazaifu, a small Japanese city located downwind of mainland East Asia. The author suggests that the observed increase in atmospheric lead concentrations in Japan are a result of long-distance transport from mainland East Asia. In addition, seasonal variation in concentrations of atmospheric lead particles was observed, with spring having the highest mean levels. This study was featured in the peer-reviewed medical journal, Biological and Pharmaceutical Bulletin, whose intended audience includes pharmaceutical scientists, pharmacologists, and regulatory professionals. 

\medskip

\noindent \bibentry{hutchinson1987lead}.

\medskip

This book compiles the findings of various researchers on the topic of lead in the environment. It provides extensive detail on the behavior and pathways of lead once emitted. In addition, the book provides estimates for the length of time that passes before lead is deposited and immobilized. There are also a number of graphics and tables in "Chapter 1 Group Report: Lead" that illustrate how and in what forms lead is deposited in our environment. The author also notes that there has been evidence of a growing number of lead sinks in Europe, Asia, South America, and North America. This has been explained by the long-distance airborne transport of lead.

\medskip

\noindent \bibentry{steding2001three}.

\medskip

This dissertation published by the University of California, Santa Cruz examines how a reduction in lead emissions has affected lead reservoirs located in the San Francisco Bay. The author’s research has shown that the phase-out of leaded gas two decades ago has had little effect on dissolved lead levels in surface waters. According to the author, unlike most environments, the reduction in lead concentrations in the San Francisco Bay has been very slow due to the complex aquatic environment.

\medskip

\noindent \bibentry{steinnes2006metal}.

\medskip

This review, published in the journal, Environmental Reviews, details the long-range atmospheric transport of lead to soils in various regions of North America and Europe. It also considers the challenge of distinguishing between naturally-caused sinks and anthropogenically-caused sinks. In addition, according to the review, anthropogenic activity has dramatically increased the atmospheric deposition of lead in Antarctica and Greenland. Lastly, it discusses the knowledge gaps that remain on the behavior of lead and other heavy metals in both the atmosphere and at the Earth’s surface.

\medskip

\noindent \bibentry{UNEP}.

\medskip

This report, created and published by the United Nations Environment Programme, provides an overview of key scientific findings for lead. I found “Chapter 7 - Long-range transport in the environment” to be especially helpful for my research on lead sinks. According to this chapter, several factors influence the distance travelled before deposition occurs including particle size, stack height, and meteorology. The report also has sections on regional-scale transboundary pollution, as well as intercontinental atmospheric transport.

\medskip

\noindent \bibentry{watmough2007lead}.

\medskip

This article, published in the international journal, Biogeochemistry, details the transport and mobility of lead in watersheds. The study concludes that deposited lead is retained well in organic soils and lake sediments. However, lead concentrations have decreased since 1989-1991, given the decrease in lead emissions. This coincides with the findings of many other studies done on the changes in lead concentrations over time. 

\medskip

\noindent \bibentry{zhang2012effect}.

\medskip

This article, featured in the 2012 International Conference on Biomedical Engineering and Biotechnology, was primarily made to inform researchers in the fields of biomedical engineering and biotechnology. The article discusses how temperature, salinity, and pH affect lead’s adsorption on sediment. The author backs up the case study’s findings by referencing the findings of various past studies. Both temperature and salinity are positively correlated with lead adsorption. As pH increased from 1 to 4, lead adsorption increased. But, as pH increased from 4 to 7, lead adsorption decreased. The results of other studies vary with this result, but on average for other studies, maximum adsorption takes place at a pH of 6. 

\medskip

\section{Food web dynamics (Mireya)}

\section{Toxicity (non-human) (Kelli)}

\section{Human health effects (physiological, toxicity) (Thea)}

\section{Public health effects (crime, IQ, etc) (Marissa)}



\end{document}
