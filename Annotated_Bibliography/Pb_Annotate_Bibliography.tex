\documentclass{article}
\usepackage{natbib}
\usepackage{hyperref}
\usepackage{bibentry}
\nobibliography*

\setcitestyle{open={(},close={)}}


\author{EA30 -- Spring 2017}
\title{Annotated Bibliography}


\begin{document}
\maketitle

\nobibliography{Pb_LiteratureCOMPLETE}


\section{Abstract}



\tableofcontents

\section{Instructions}

Please complete the annotated bibliography by the 17th of April.

\subsection{Searching for Academic Sources}

There are numerous electronic sources to evaluate and obtain academic sources of information. Usually, we consider peer reviewed articles to have the highest quality scholarship. As a rule of thumb, this is a good start, but there is also a great deal of variation between sources, even journal titles can vary in quality. Thus, it's best to evaluate a range of sources and appreciate the subtle differences in quality and prestige.

First, you should search CUC databases using key words, such as Pb and lead. I admit the lead is tricky, because it's not a useful term due it a double meaning. I suggest heavy metals or trace metals with the other key words, i.e. fate and transport, toxicity, etc, depending on the topic you selected.

Below are links to some useful databases:

\begin{itemize}
  \item \href{http://apps.webofknowledge.com.ccl.idm.oclc.org/}{Web of Science}
  \item \href{http://www.jstor.org.ccl.idm.oclc.org/}{JStore}
  \item \href{https://scholar-google-com.ccl.idm.oclc.org/}{Google Scholar}
\end{itemize}

\subsection{Writing an Annotated Bibliography}

After adding your citation to the Pb\_literature.bib, using the BibTeX format, cite your reference using the syntax below. Then, write a concise annotation that summarizes the central theme and scope of the book or article. Include one or more sentences that (a) evaluate the authority or background of the author, (b) comment on the intended audience, (c) compare or contrast this work with another you have cited, or (d) explain how this work illuminates your bibliography topic.

\subsection{Implementing in \LaTeX}

\LaTeX is a software program used for desktop publishing. With an eye for detail, the program was developed to give the author a great deal of control. In contrast, Microsoft Word is designed to have lots of options, but these seem to get in the way of controlling the outcome. 

\LaTeX is an open source program and relies on specific formatting commands that begin with a backslash. For example to start a new section heading, we use \verb!\section{section name}! and \verb!\subsection{subsection name}! to create a subsection heading.

Below might be an example of what we are trying to accomplish using our Rstudio resources. The \LaTeX command for the citation is \verb!\bibentry{key}!, where key is identifies the citation based on the BibTeX citation entry as shown below.

However, we also need to add the citation to the *.bib file, in this case the file is called ``Pb\_literature.bib''. Open the file. You can see each entry within the curly brackets. Also, it starts with the definition of the entry, e.g. article, book, conference proceedings. We can easily add the citation using the databases cited above. All you need to do is search for the citation manager or citation tool and select a citation format ``bibtex''. Copy and paste the bibtex format (try to put it in the correct location, i.e. alphabetic order). Note the ``key'' is the first word after the citation definition.

\subsection{Peculiarities of \LaTeX}

Some characters are not automatically recognized in \LaTeX. For example, the \& (ambersand) and \_ (underscore) generate errors because they are reserved for different formatting functions. To use them as text characters, they need to be preceeded by a backslash, \, i.e. \verb!\&! and \verb!\_!.  We can also put the sources as bullets to help emphasize the citation.+

\subsection{Annotated Example}

\bigskip
\noindent \bibentry{Barltrop1975absorption}.  Summarizes important infomormation about Pb. This paper reviews work that span x years starting with early \ldots and noting recent information that includes \ldots. \cite{Barltrop1975absorption} used XX methods to...

\section{Industrial Sources (Khalil)}

\bibentry{singh1997atmospheric}. sss

\bibentry{harrison2012lead}.

\bibentry{stigliani1993heavy}.

\bibentry{nriagu1996history}.

\bibentry{rieuwerts1996heavy}. 

\bibentry{duzgoren2007sources}. 

\section{Use in Industry (except gasoline) (Claudia)}

\bibentry{Beris2003}

\bibentry{Hodgkins1992}

\medskip
\bibentry{Gibson1968}. A series of 100 lead workers from different industries, 91 at work and nine admitted to hospital with lead poisoning, was studied in order to define more clearly the clinical and biochemical criteria of lead poisoning in three stages: A) a presymptomatic state of lead exposure (37 men), (B) a state of mild symptoms or mild anemia (45 men), and (C) frank lead poisoning with severe symptoms and signs (18 men). The tests used were hemoglobin, reticulocyte count, and blood lead, and urinary lead, coproporphyrin, ?-aminolaevulinic acid (ALA), and porphobilinogen (PBG) estimations. Of these, the urinary lead was similar for all three groups and the blood lead estimation was of less value for determining the clinical group of the men than the hemoglobin and urinary coproporphyrin or ALA estimations, which correlated well with the clinical assessment and with each other but showed no correlation with the urinary and blood lead levels. PBG levels became raised only with the onset of symptoms of lead poisoning. A hemoglobin of 13 g./100 ml. (90\%) or less is a cautionary sign. Urinary coproporphyrin above 80 ?g./100 mg. creatinine (800 ?g./liter), ALA above 2? mg./100 mg. creatinine (2? mg\%), and PBG above 0?5 mg./100 mg. creatinine (0?5 mg.\%) were almost always associated with symptoms or signs and were therefore considered to be the upper safety limits. Although the blood lead level does not differentiate between lead toxicity and lead exposure, values above 60 ?g. lead/100 g. blood should alert the physician to carry out other tests. In addition to the above tests, blood pressure, blood urea, and serum uric acid estimations were performed on all the men in order to elucidate the possible role of lead in the production of renal damage. Blood pressure and serum uric acid levels were similar for all three groups but the blood urea level was raised in group C. The reason for this finding was not established. It was found that scrap metal burning, battery manufacturing, and ship-breaking constituted the gravest lead hazards encountered in this survey whereas wire manufacture constituted the least. Workers in the most modern factory, a car-body pressing plant, gave average values just below the danger levels for the urinary coproporphyrin and ALA estimations despite apparently efficient protective measures. This finding underlines the importance of the medical supervision of lead workers.

\bibentry{Lanphear1998}.

\medskip

\bibentry{Ernhart426}. 

\medskip

\bibentry{schwartz2000past}.

\medskip

\section{``Ethyl'' gas and airplane fuels (Viraj)}

\noindent \bibentry{americanoil} .

The American Oil and Gas Historical Society offers official history of energy fuels for increasing public knowledge. Tetraethyl Lead was added to gasoline to prevent a common problem in engines before the 1930s called “engine knock.” Engine knock occurs when the air-fuel mixture in a cylinder about the piston prematurely detonates before the piston has risen to the maximum height. This premature detonation occurs because of the gasoline not being able to withstand the pressure. In the ideal internal combustion engine operation, the air-fuel mixture should only combust because of a spark created by the spark plug, not an increase in pressure. A gas’s ability to withstand pressure is defined by its octane rating. Adding tetraethyl lead to gasoline increase the octane rating, makes the fuel more resistant to combusting under pressure, and therefore prevents engine knock. In 1921, General Motors scientists discovered the anti-knock properties of tetraethyl lead. Three years later, the first signs that leaded gasolines contributed to lead poisoning occurred when workers at a gasoline treatment plant fell ill. In 1925, after minimal non-comprehensive studies, the U.S Surgeon General stated that there is “no reason to prohibit the sale of leaded gasoline.” The leaded gasoline industry used many manipulative techniques to hide the fact that lead was present in gasoline. They labelled the higher octane fuel “ethyl” gasoline, to avoid lead in the name. The word lead was never used in any advertisements for the fuel. Ultimately, in the late 1970s lead was phased out of gasoline and in 1986 a total ban was placed on leaded automobile gasoline. The invention of the catalytic converter in 1975, which required unleaded fuel, also accelerated the phasing out of lead from gasoline. 

\bigskip


\noindent \bibentry{exxonmobil} .

Exxon Mobil is a manufacturer and an authority on fuels and energy products based in petroleum and natural gas. Jet Fuel is different from other fuel for two main reasons. The flash-point for jet fuel is much higher than other fuels. This is for safety reasons because jet fuel should not spontaneously combust because there is frequently a large volume of it near many people in the setting of a commercial airliner. When fuel gets extremely cold at low temperatures gas interacts with freezing water and forms a substance in the fuel called “wax.” This wax lowers the quality of the fuel. Since planes fly at high altitudes where it is extremely cold, airplanes add fuel to lower the freezing temperature of the fuel as well. Jet fuel A and Jet fuel A1 are both unleaded kerosene. Jet Fuel is similar to diesel fuel and can be used in either compression ignition engines or turbine engines.The primary difference between the two is freeze point, the temperature at which wax crystals disappear in a laboratory test. Jet A, which is mainly used in the United States, must have a freeze point of minus 40ºC or below and does not typically contain static dissipator additive. Jet A-1 must have a freeze point of minus 47ºC or below and for locations outside the United States, this fuel normally contains static dissipator additive. There are other key differences between the manufacturing specification within the United States and Europe/Africa/Middle East/Australasia.Ultimately, jet fuel used by large commercial airliners and helicopters  is not an issue in term of emitting lead into the atmosphere.

\bigskip

\noindent \bibentry{FAA} .

The Federal Aviation Administration (FAA) monitors fuel regulation and fuel use in planes. It is therefore a credible authority providing information about current use of Aviation Gas (avgas). The FAA shares the Environmental Protection Agency's (EPA) concerns about lead emissions from small aircraft. Owners and operators of more than 167,000 piston-engine aircraft operating in the United States rely on avgas to power their aircraft. Avgas is the only remaining lead-containing transportation fuel. The FAA has begun slowly taking steps to phase out leaded aviation fuel. Via the Piston Aviation Fuels Initiative (PAFI), the FAA has begun testing two unleaded fuel alternatives in 2016, and has said that all testing will be finished by the end of 2018. Some other sources said that they would have an alternative fuel by 2018. There is no guarantee that an alternative fuel will be approved by 2018. 


\bigskip
\noindent \bibentry{galzigna1973biochemical} .

This study by Galzigna et al. is published in the British Journal of Industrial Medicine. The paper speaks to the negative health effects that result from exposure to tetraethyl lead. This experiment tested the effects of tetraethyl lead on rats. The effects of tetraethyl lead are similar to general lead exposure. The primary negative health effect observed was reduced neuromuscular function.  

\bigskip


\noindent \bibentry{miranda2011geospatial} .

Miranda et al. at Duke University in 2011 found that kids living within 500 meters of an airport where leaded avgas is used have higher blood lead levels than other children, with elevated lead levels in blood found in kids as far as one kilometer away. The researchers estimate a “significant association between potential exposure to lead emissions from avgas and blood lead levels in children.” The findings in this study place pressure on the FAA to look for replacing leaded gasoline in small piston engine planes. 

\bigskip

\noindent \bibentry{rosner1985gift} .

Author David Rosner is the Ronald H. Lauterstein Professor of Sociomedical Sciences and Professor of History in the Graduate School of Arts and Sciences at Columbia University. Rosner outlines the history of leaded gasoline and the rise of the petrochemicals industry in the mid 1920s to the increased public awareness of the negative health effects of leaded gasoline in the 1970s. Rosner compellingly focuses on many of the propaganda efforts undertaken by the leaded gas industry to sell their gas to people and mask the lead present in "Ethyl" gas. 

\bigskip

\noindent \bibentry{scheermoss2012} .

In this article, Roddy Scheer and Doug Moss at the Scientific American investigate and synthesize a 2011 Duke University Study to show the impact of avgas on lead level. The EPA estimates that 16 million Americans live close to one of 22,000 airports where leaded avgas is routinely used—and three million children go to schools near these airports. While jet fuel used by large commercial airlines is unleaded, the fuel used by smaller passenger planes with piston engines is still leaded. This leaded gasoline is known as Aviation Gas or Avgas. Since the phase out of lead in automobile fuel in the 1970s, aviation fuel emerged as the largest source of lead emissions in the U.S. Aviation fuel accounts for half of the lead pollution in American skies, making it a real air quality issue. 


\bigskip

\noindent \bibentry{shellglobal} .

Royal Dutch Shell Corporation produces Avgas and is an authority on its use. Crop spraying planes run on leaded avgas. This is a problem considering that lead emitting planes are operating with close proximity to the food humans consume. The 2011 Duke study cited above also speaks to the danger of proximity to the lead emmitting source.

\bigskip



\bigskip
\noindent \bibentry{galzigna1973biochemical}.
\medskip

This study by Galzigna et al. is published in the British Journal of Industrial Medicine. The paper speaks to the negative health effects that result from exposure to tetraethyl lead. The effects of tetraethyl lead are similar to general lead exposure. The primary negative health effect observed was reduced neuromuscular function.



\section{Atmospheric transport and deposition}
\begin{itemize}
  \item \bibentry{steinnes1997evidence}. \cite{steinnes1997evidence} documents soil contamination in Norway. Depending on the metal, the source, i.e. the relative contribution from local and long-range sources vary. 
  \item \bibentry{cohan2010potential}
  \item \bibentry{blais1996using}
  \item \bibentry{cortizas2002atmospheric}
\end{itemize}


\section{Aquatic transport (Olivia)}

\noindent \bibentry{alleman1999invasion}.
\medskip

This study analyzes stable lead isotopic composition data to reveal the advective transport of industrial lead into deep basic waters through the formation of North Atlantic Deep Water. The authors analyzed sea water samples collected during a suite of cruises in the far North Atlantic, western North Atlantic, North African basins, and equatorial North Atlantic. The samples showed that 206Pb/207Pb ratios of newly formed North Atlantic Deep Water in the far North Atlantic reflect mixing with less radiogenic western European emissions owing to the phasing out of leaded gasoline in the US. The authors predict that in future, stable lead isotopes will continue to uniquely fingerprint geographic sources of lead and associated contaminants, and trace transient fluxes of those contaminants into the abyssal North Atlantic over decadal scales.

\medskip
\noindent \bibentry{bellinger2016lead}.
\medskip

This article was published in the New England Journal of Medicine by Dr. David C. Bellinger, a Professor of Neurology at Harvard Medical School and Professor in the Department of Environment Health at Harvard T.H. Chan School of Public Health. He writes about historical and modern lead exposure, particularly through water distribution systems. He then explains the institutional inequalities surrounding Flint's water contamination and the effects on public health. 

\medskip
\noindent \bibentry{echegoyen2014recent}.
\medskip

This study of northern and central Indian Ocean found high concentrations of Pb in surface waters due to rapid regional industrialization and low concentrations in deep waters. The Antarctic sector of the Indian Ocean shows very low Pb concentrations due to limited anthropogenic emissions, high scavenging rates, and rapid vertical mixing. The article predicts an increasing presence of Pb in the ocean following the continued release of Pb from human activities. The data was collected from 11 stations in the 2009-2010 Japanese GEOTRACES transect through the Indian Ocean. The authors of this study are from either Massachusetts Institute of Technology, The University of Tokyo, and Niigata University. 

\medskip
\noindent \bibentry{gobeil2001atlantic}.
\medskip

This study published by the American Association for the Advancement of Science examines sediment cores collected from the Arctic Ocean and Greenland Sea evidence for evidence of lead. The researchers then used the Pb inventory and its isotopic composition to infer the source of the lead contamination and its pathways related to ocean currents and ice drift. The authors are affilliated with Institut Maurice-Lamontagne, Institute of Ocean Sciences, and Bedford Institute of Oceanography in Canada.

\medskip
\noindent \bibentry{hem1973solubility}.
\bigskip

This article evaluates the solubility of lead in solutions similar to natural river water to determine the general features of water chemistry, which features favor the solution of lead, and the solubility effectiveness. The data in this article also suggest methods of reducing the lead content of drinking water at water-treatment plants. 

\medskip
\noindent \bibentry{kim2000factors}.
\medskip

This study measured the atmospheric depositional fluxes of 7Be, 210Pb, and stable Pb simultaneously for one year on the upper eastern shore of the Chesapeake Bay in Maryland. The research suggests that a constant Pb flux has been reached since the mandatory use of unleaded gasoline was instituted. This study concludes that precipitation appears to be an important factor controlling the fluxes of 7Be and stable Pb in the upper troposphere. However, 210Pb in the lower troposphere is highly scavenged from the atmosphere from small amounts of precipitation such as snowfall. 

\medskip
\noindent \bibentry{nozaki1976distribution}.
\medskip

This study determines residence times relative to particulate removal or 210Pb and 210Po in surface waters of the Pacific Ocean in order to examine the transportation of trace metals in the ocean column. The authors collected surface water samples with two 2-liter plastic sample bottles enclosed in a net attached to a ship moving at low speed. The study found that the residence times for Po and Pb in the center of the North Pacific gyre are 0.6 years and 1.7 years, respectively. The fact that surface ocean residence time for Pb is about two orders of magnitude smaller than for deep ocean Pb implies that adsorptive quality of particles changes sharply during descent through the water column.

\medskip
\noindent \bibentry{schock1980response}.
\medskip

Michael Schock works with the United States Environmental Protection Agency and has written extensively on lead corrosion and drinking water contamination. His paper presents a revised model showing the response of theoretical lead solubility curves to changes in dissolved inorganic carbonate concentrate and pH of water at 25 degrees Celsius. The author acknowledges that his model cannot be directly applied to accurately predict concentrations of lead in tap water in many situations because of the complex chemical and physical concentration control mechanisms happening in the water distribution system. However, the model improves upon earlier models that did not take into account hydrocerussite and lead carbonate complexes. The revised model had good correlation with experimental data and can therefore be employed with some confidence to measure lead solubility.

\medskip
\noindent \bibentry{stansley1992lead}.
\medskip

This paper reports lead contamination at shooting ranges where shots have been deposited in open water or wetland areas and the impacts on a nearby lake. The authors found a significant concentration of filterable lead in the slightly acidic marsh, which suggests that lead could be mobilized at a lower pH. There were elevated lead levels in a  water sample collected near a public swimming area adjacent to a parking lot, possibly as a result of parking lot runoff. The report found elevated total lead concentrations in surface water in the shot fall zones at six trap and skeet ranges, but found no lead contamination in water, sediments, or fish from a lake adjacent to the range with the greatest amount of shot deposition. However, no measurements of particle size or organic content were performed, both of which have a strong influence on sediment lead concentrations. The sample size of fish studied for uptake of lead was also small. This study was conducted by three researchers from the New Jersey Divsion of Fish, Game and Wildlife. 

\medskip
\noindent \bibentry{windom1985geochemistry}.
\medskip

This study was conducted in order to establish baseline levels of lead in coastal and estuarine waters of the southeastern United States and evaluate the relative importance of inputs from rivers, the atmosphere, and oceanic exchange. It was published in 1985 by a researcher from the Skidaway Institute of Oceanography and a researcher from Tokyo Universit of Fisheries.

\medskip
\noindent \bibentry{wu1997lead}.
\medskip

This article investigates the significant decrease of lead concentrations in the western North Atlantic Ocean over a 16-year period from the 1980s to the 1990s. The authors attribute the rapid decline in the 1980s to the phasing out of leaded gasoline in the United States. They propose the slower decrease in the 1990s is due to emissions from high-temperature industrial activities rather than residual leaded gasoline emissions. They predict minimal decreases in surface water lead concentrations in coming decades, and continued propogation and evolution of lead distribution in deeper waters. The two authors of this paper are both scientists with the Department of Earth, Atmospheric, and Planetary Sciences at the Massachusetts Institute of Technology.


\section{Sinks (Katie)}

\noindent \bibentry{agency1990toxicological}.
\medskip

This toxicological profile created by the Agency for Toxic Substances \& Disease Registry and the Environmental Protection Agency details the human health effects of lead exposure. It also provides information on the behavior of lead in soil and sediment and the factors that influence what areas become sinks. For example, temperature, pH, and the presence of humic materials are highly important in determining the mobility of lead in soil. The intended audience for this document are public health professionals, interested private organizations, and the general public. 

\medskip

\noindent \bibentry{bazzano2016long}.

\medskip

This article, featured in the peer-reviewed scientific journal, Atmospheric Environment, discusses the long-range transport of lead particulates to the Norwegian Arctic. The intended audience for this journal are scientists whose work is related to atmospheric science. The article also notes that there is a seasonal trend in lead concentrations. For instance, lead concentrations are higher in the spring and the summer. The lead accumulating in the Norwegian Arctic during the summer is a result of North American industrial emissions. 
\medskip

\noindent \bibentry{amodio2014atmospheric}.

\medskip

This article is featured in Advances in Meteorology, a peer-reviewed, Open Access journal whose primary audience are scientists in the field of meteorology. The article details the mechanisms of atmospheric deposition of heavy metals, such as lead. In addition, the location of deposition is strongly dependent on the climate and the amount of rainfall the area receives. 

\medskip

\noindent \bibentry{coulibaly2015seasonal}.

\medskip

This study, supported by the Japanese Ministry of the Environment, focuses on the effects of long-range transport of lead and other pollutants on Dazaifu, a small Japanese city located downwind of mainland East Asia. The author suggests that the observed increase in atmospheric lead concentrations in Japan are a result of long-distance transport from mainland East Asia. In addition, seasonal variation in concentrations of atmospheric lead particles was observed, with spring having the highest mean levels. This study was featured in the peer-reviewed medical journal, Biological and Pharmaceutical Bulletin, whose intended audience includes pharmaceutical scientists, pharmacologists, and regulatory professionals. 

\medskip

\noindent \bibentry{hutchinson1987lead}.

\medskip

This book compiles the findings of various researchers on the topic of lead in the environment. It provides extensive detail on the behavior and pathways of lead once emitted. In addition, the book provides estimates for the length of time that passes before lead is deposited and immobilized. There are also a number of graphics and tables in "Chapter 1 Group Report: Lead" that illustrate how and in what forms lead is deposited in our environment. The author also notes that there has been evidence of a growing number of lead sinks in Europe, Asia, South America, and North America. This has been explained by the long-distance airborne transport of lead.

\medskip

\noindent \bibentry{steding2001three}.

\medskip

This dissertation published by the University of California, Santa Cruz examines how a reduction in lead emissions has affected lead reservoirs located in the San Francisco Bay. The author’s research has shown that the phase-out of leaded gas two decades ago has had little effect on dissolved lead levels in surface waters. According to the author, unlike most environments, the reduction in lead concentrations in the San Francisco Bay has been very slow due to the complex aquatic environment.

\medskip

\noindent \bibentry{steinnes2006metal}.

\medskip

This review, published in the journal, Environmental Reviews, details the long-range atmospheric transport of lead to soils in various regions of North America and Europe. It also considers the challenge of distinguishing between naturally-caused sinks and anthropogenically-caused sinks. In addition, according to the review, anthropogenic activity has dramatically increased the atmospheric deposition of lead in Antarctica and Greenland. Lastly, it discusses the knowledge gaps that remain on the behavior of lead and other heavy metals in both the atmosphere and at the Earth’s surface.

\medskip

\noindent \bibentry{UNEP}.

\medskip

This report, created and published by the United Nations Environment Programme, provides an overview of key scientific findings for lead. I found ``Chapter 7 - Long-range transport in the environment''
 to be especially helpful for my research on lead sinks. According to this chapter, several factors influence the distance traveled before deposition occurs including particle size, stack height, and meteorology. The report also has sections on regional-scale transboundary pollution, as well as intercontinental atmospheric transport.

\medskip

\noindent \bibentry{watmough2007lead}.

\medskip

This article, published in the international journal, Biogeochemistry, details the transport and mobility of lead in watersheds. The study concludes that deposited lead is retained well in organic soils and lake sediments. However, lead concentrations have decreased since 1989-1991, given the decrease in lead emissions. This coincides with the findings of many other studies done on the changes in lead concentrations over time. 

\medskip

\noindent \bibentry{zhang2012effect}.

\medskip

This article, featured in the 2012 International Conference on Biomedical Engineering and Biotechnology, was primarily made to inform researchers in the fields of biomedical engineering and biotechnology. The article discusses how temperature, salinity, and pH affect lead’s adsorption on sediment. The author backs up the case study’s findings by referencing the findings of various past studies. Both temperature and salinity are positively correlated with lead adsorption. As pH increased from 1 to 4, lead adsorption increased. But, as pH increased from 4 to 7, lead adsorption decreased. The results of other studies vary with this result, but on average for other studies, maximum adsorption takes place at a pH of 6. 

\medskip

\section{Food web dynamics (Mireya)}

\bibentry{cardwell_cd_2013}.

\bigskip

This study used empirical laboratory data, modeling, and field studies to determine whether lead is biomagnified in aquatic ecosystems. Cardwell et al. concluded that lead does not biomagnify through food chains consisting of primary producers, macroinvertebrates, and fish higher than trophic-level three. This study offers information contrary to that of Alexander et al., and furthers the debate regarding lead’s biomagnification potential. 

\bibentry{laskowski_are_1991}.

\bigskip

Laskowski’s paper, published in Oikos on behalf of the Department of Ecosystem Studies at Jagiellonian University, argues that biomagnification is not a shared property of all terrestrial ecosystems but rather something that varies dramatically based on an organism’s ability to process the substance.  The paper dismisses the role of biomagnification in poisoning larger mammals, claiming that more than three or four trophic levels are needed to accumulate a dangerous amount. I thought the article’s points on the subjectivity of biomagnfication was interesting, although I felt some of the larger conclusions were not grounded in enough evidence and were made hastily at the end. 

\bibentry{pinto_heavy_2003}.

\bigskip

Published in the Journal of Phycology, this paper finds that high levels of metal pollutants attack the chloroplast antioxidants in algae cells, eventually leading to their death. Algae are primary producers in the aquatic food chain and produce most the ocean’s oxygen; if algae are unable to perform their functions properly, a trophic cascade could wreak havoc on the entire food web.  

\bigskip

\bibentry{chen_accumulation_2000}.

\bigskip

This study, published in The Journal of Limnology and Oceanography, measured heavy metal concentrations in water, plankton, and fish from 20 lakes in watersheds across the United States. The study evaluates various factors that affect lead concentration in bodies of water, such as type of land use, water temperature, pH, etc. Notably, it stresses the need to evaluate higher and lower trophic levels separately, rather than attempt to understand them as the same metabolic mechanisms. The paper found that lead biodiminshed with trophic level, meaning bioaccumulation decreased with each energy transfer. 

\bibentry{dmello_food_2003}.

\bigskip

Chapter 9 in the book is title Heavy Metals and has a subsection on lead. The book offers some useful statistics, like how adults uptake 10\% of Pb from food, and children 50\% (p. 211). It also touches upon the success of source-related lead phase-outs in food within the past decades. It will prove useful when evaluating how lead is ingested and absorbed by humans. 

\bigskip

\bibentry{islam_assessing_2007}.

\bigskip
Published in the Journal of Zhejiang University, this study evaluates the concentration of lead in agricultural crops. The study identifies multiple factors that may affect bioavailability of heavy metals in crops, such as fertilizer and crop rotation, while also highlighting the influence of root microbes on heavy metal uptake. The paper will be a useful in analyzing how lead enters human food systems and how agricultural practices can improve to prevent lead uptake. 
\bigskip

\bibentry{alexander_bioaccumulation_1999}.
Published in the Encyclopedia of Earth Science, this paper provides essential definitions for bioaccumulation, bioconcentration, and biomagnification. The paper is written for a scientific journal and employs a lot of biological terminology that may be inaccessible to some. The paper explains that heavy metals, such as lead, tend to bioaccumulate and then biomagnify because they pass through the lipid bilayer and cannot be easily dispelled by the body. 


\section{Toxicity (non-human) (Kelli)}

\bibentry{sharma2005lead}. 
\bigskip

Pallavi Sharma is a biochemistry and molecular biology professor at the Central University of Jharkhand. She holds both a Master's and PhD in biochemistry, and she has published 28 articles relating to heavy metal toxicity in plants. The article cited here in particular was published in 2005, and has been cited by 887 articles since then. 
This article discusses the science of Pb uptake in plants as well as Pb's potential effects on plant growth and health. The article appears to be geared towards academics—but its points are still generally accessible to readers such as college students. 

\bigskip 

\bibentry{fisher1996accumulation}.
\bigskip

Nicholas S. Fisher is a Distinguished Professor at Stony Brook University, specializing in topics such as marine pollution and the biogeochemical cycling of metals in marine environments. He has published over 200 articles on these topics, and the cited here in particular has been cited 159 times since it was published in 1996. The article is written very technically for an audience that has a strong scientific background. This article explained the author's findings that metals such as Pb tend to accumulate in shells of mussels rather than in their soft tissue, as well as the presence of Pb in mussel fecal matter that could contribute to its biogeocycling in the environment.  

\bigskip 

\bibentry{weber1997alterations}.
\bigskip

Daniel N. Weber is a professor at the University of Wisconsin in Milwaukee, whose research is centered around the effects of metal contaminants such as lead on fish behavior. He has written 43 articles on the topic, and the particular article cited above has been cited 58 times since it was published in 1997. The article is extremely scientific and clearly written for a scientific audience. The article discusses the behavioral dysfunction that occurs in fish as a result of constant exposure to Pb, often interfering with their ability to react to stimuli. 

\bigskip 

\bibentry{burger1995risk}.

Joanna Burger is a professor at Rutgers University whose research often concerns environmental pollution and behavioral ecology, with special focus on anthropogenic sources of metals and their effects on birds. She has written 84 articles on these topics, with the one cited above cited 81 times since it was published in 1995. While other articles were definitely written with technical scientific terminology, this article is more accessible to a wider audience as it discusses the effects of Pb bioaccumulating in birds. According to Burger, as in fish discussed in Weber's article, Pb in birds also leads to hindered reactions or responses to stimuli. It also leads to developmental defects in clutches and eggs. 

\bigskip

\bibentry{mautino1997lead}.

\bigskip

Michele Mautino is a doctor of veterinary medicine in working at the zoo in Jacksonville, Florida. She once worked as a vet at Disney World and even helped to design their Animal Kingdom zoo hospital. Her research focuses on metal intoxication in zoological medicine. Her article cited above has been cited 18 times since 1997. The article is written for a scientific audience. It explains how lead in mammals is known to disrupt genetic processes as well as neurological functions—an effect similarly discussed in previous articles.

\section{Human health effects (physiological, toxicity) (Thea)}

\bigskip

\bibentry{goyer1990environmental}. 

\bigskip

This article, published in 1990, provides its readers with an objective overview of the human health effects of lead with a focus on occupational lead exposure. The article focuses on identifying and communicating the methods for toxicologists and epidemiologists to determine the lowest level of lead exposure that is toxic and may result in an array of harmful health effects. In discussing the various health effects of lead exposure, Goyer cites multiple studies, involving both samples of lead workers and the general public, that have shown a relationship between lead exposure and an increase in blood pressure, an increase in hypertension, a number of minor congenital malformations, the hinderance of skeletal growth, impairment to the kidney, and impairment to vitamin D metabolism. This article contains specific scientific vocabulary when describing the different ways in which lead affects a variety of organ systems, which suggests that this article is intended for an audience with a strong understanding of human biological processes. Therefore, without the help of outside research, this article is harder to fully comprehend when compared to my other citations. Unfortunately, the author, Goyer, passed away. He has written over 150 research papers on the toxicity of lead and other metals that have been cited hundreds of times. Additionally, Goyer served on committees for US and International Health Agencies including the National Institutes of Health, the Environmental Protection Agency, and The World Health Organization International Program for Chemical Safety and was recognized by the National Academy of Sciences in 2001 for his service as an advisor in matters of science, engineering and health. I therefore believe Goyer is a trustworthy source and authority on this topic. This article was published in the journal Environmental Health Perspectives, a collection of scientific articles and discussion focused on research concerned with toxicity and approaches to detecting and mitigating environmental damage. 

\bigskip


\bibentry{hammond1977annual}. 

\bigskip
Paul Hammond’s article, “Exposure of humans to lead” provides a comprehensive overview of the various human health effects to lead. Specifically, Hammond’s breaks up his article into multiple sections that focus on the uptake of lead by environmental sources, the absorption and distribution of lead into the body, and the effects on different organ systems.  While many of my other sources primarily focus on the various human health effects of lead, this article is especially useful for its explanations on how lead is able to enter and spread through the body through its descriptions of lead uptake and absorption by the skin and gastrointestinal tract. Still, Hammond provides comprehensive details on the effects of lead exposure on the body through his focus on the reactions of heme and hemoproteins and the kidney to lead exposure.  This article was published in Annual Reviews, a large nonprofit publisher of scientific research that is meant for both scientists and students alike. This article has been cited 70 times and Paul Hammond has done research at the Department of Environmental Health at the University of Cincinnati Medical Center, and is therefore a relatively trustworthy and authoritative source.

\bigskip
\bibentry{juberg1997position}. 

\bigskip
This article, provided by the American Council on Science and Health, details various components on human exposure to lead and the proceeding effects. This article is divided into eight highly informative sections on the course of lead exposure, its consequences, and possible strategies for mitigating such effects. These sections include lead in the environment, human exposure, the toxicology of lead, lead in consumer products, regulatory initiatives for limiting exposure to lead, what is a safe level of lead, and lead abatement, and a conclusion that includes recommendations that consist of both personal and public strategies for minimizing lead exposure.  The authors, while nonetheless providing comprehensive descriptions on the complex relationship between human health and lead exposure, use more accessible language that allows for a wider audience to understand this issue. This article describes the effect of lead-exposure on various organ system and thus details lead’s links to kidney failure, bone development, vitamin D metabolism, the reproductive system, blood pressure, child development, and the possibility of links to cancer.  Additionally, this article includes helpful charts and figures to guide the reader’s understanding these relationships. The American Council on Science and Health, a science education nonprofit with the mission of supporting evidence-based science and medicine, published this article in the journal “Ecotoxicology and Environmental Safety”. Due to its long list of accredited peer reviewers, the credentials of its authors, and its publisher, I find this article to be authoritative and trustworthy. 

\bigskip

\bibentry{WHO2016lead}. 

\bigskip
This website summarizes the key takeaways on the relationship between lead exposure and human health. Although not the most comprehensive source, the site both provides important background information for those who are new to the subject and uses very accessible language resulting in digestible information for the general public. This website focuses on the sources of lead exposure, its effects on children, the burden of disease from lead exposure, and the response of the World Health Organization (WHO). WHO, an agency of the United Nations that is focused on international public health, has identified lead as 1 of 10 chemicals of major public health concern. This site is my only source that gives more broad details on the global impact of health risks due to lead exposure. For example, this site explains that the Institute for Health Metrics and Evaluation (IHME) has estimated that in 2013 lead exposure accounted for 853 000 deaths due to long-term effects on health. WHO is doing work on the prevention and management of lead exposure by drafting guidelines to provide policymakers, public health authorities, and health professionals with evidence-based guidance on measures that they can take to protect the health of children and adults from lead exposure. I therefore believe WHO to be a credible and authoritative source. 

\bigskip

\bibentry{agency2016lead}. 

\bigskip
This website was published by the Agency for Toxic Substances and Medicine Exposure, a federal public health agency of the U.S. Department of Health and Human Services that works to provide information on harmful exposure and disease related to toxic substances. This particular section on their website is part of a larger educational course on lead toxicity meant for people working in the health sector who may work with patients exposed to lead. The page that I have cited focuses on answering the question, what are the physiologic effects of lead exposure? The learning objectives for this page include being able to describe how lead affects adults and children, describe the major physiologic effects of chronic low-level lead exposure, and describe the major physiologic effects of acute high-level lead exposure. The page begins by describing how the effects of lead exposure differ between children and adults and then is broken down into multiple sections that focus on the effect of lead on different organs and systems. Because this site is meant for those working within the health sector, there is more use of challenging medical and biological language than in my other sources. Nonetheless, this source is very helpful in its concise and in depth descriptions on how lead exposure may affect different organ systems. 

\bigskip

\bibentry{needleman1991health}. 

\bigskip

This article focuses on the health effects of low level lead exposure for children. Through in-depth analysis of various studies, the authors, Needleman and Bellinger, describe the neurological effects of lead exposure on infants and children. These effects include lower IQs and behavioral changes, such as the reduction of attention span. Additionally this article discusses the effects of lead exposure on infant and child development, siting correlations between malformations, organ impairment, and hinderance of physical growth to lead blood levels. Throughout the article, the authors provide multiple graphs and figures that demonstrate these findings and help to synthesize the information. Herbert Needleman is well-known for his research on neurodevelopmental damage caused by lead-poisoning and David Bellinger is currently a professor of neurology at Harvard Medical School and a professor in the department of Environmental Health at the Harvard School of Public Health. Due to the authors’ credentials and their article being cited over 500 times I believe the authors to be authoritative on this subject and for this article to be a trustworthy source. 

\section{Public health effects (crime, IQ, etc) (Marissa)}

\noindent \bibentry{Hou2013}. 

\medskip

In this study, \cite{Hou2013} investigated the relationship between lead poisoning and the intellectual and neurobehavioral capabilities of children. The background characteristics of the research subjects were collected by questionnaire survey. Intelligence was assessed using the Gesell Developmental Scale. The Achenbach Child Behavior Checklist was used to evaluate each child’s behavior. In result, blood lead levels were significantly negatively correlated with the developmental quotients of adaptive behavior, gross motor performance, fine motor performance, language development, and individual social behavior (P < 0.01). Compared with healthy children, more children with lead poisoning had abnormal behaviors, especially social withdrawal, depression, and atypical body movements, aggressions and destruction. This is consistent with other studies’ findings for adverse effects on intellectual development in children. Connecting the neurobehavioral capabilities of children with intellect sets the groundwork for the issues of public health caused by lead poisoning.

\medskip

\noindent \bibentry{miranda2007}. 

\medskip

This study by \cite{miranda2007} implies that blood lead levels of 20–50 μg/L can impair the reading and math abilities of children. This was found through testing performance on end-of-grade testing for 4th-grade students from 2000-2004 North Carolina Education Research Data Center. Ultimately, the discernible impact of blood lead levels on end-of-grade testing is demonstrated for early childhood blood lead levels as low as 2 μg/dL. A blood lead level of 5 μg/dL was associated with a decline in end-of-grade reading and mathematics scores that is roughly equal to 15 percent (14 percent) of the interquartile range. This impact is very significant in comparison with the effects of covariates typically considered profoundly influential on educational outcomes. Specifically, early childhood lead exposures appear to have more impact on performance on the reading than on the mathematics portions of the tests. These results suggest that the relationship between blood lead levels and cognitive outcomes are robust across outcome measures and at low levels of lead exposure. This study holds importance for public health because it specifies the types of intellectual limitations being created by environmental injustice. 

\medskip

\noindent \bibentry{needleman_1984}. 

\medskip

This study by \cite{needleman_1984} tested umbilical cord blood from 5,183 consecutive deliveries of at least 20 weeks' gestation and analyzed them for lead concentration. Demographic and socioeconomic variables, such as lead, which were shown on univariate analysis to be associated with increased risk for congenital anomalies were evaluated and controlled by entering them into a stepwise logistic-regression model with malformation as the outcome. Coffee, alcohol, tobacco, and marijuana use, which were associated with lead level, but not risk of malformation, were also controlled. Lead was found to be associated, in a dose-related fashion, with an increased risk for minor anomalies. This study is widely used and heavily cited. This paper successfully frames the intersection of lead poisoning and public health by speaking to readers of diverse backgrounds through accessible language.

\medskip

\noindent \bibentry{nevin2000lead}.

\medskip

Rick Nevins was one of the first to dive into the correlation between crime and lead. This study concludes that lead emissions from automobiles explain 90 percent of the variation in violent crime in America. Also, toddlers who ingested high levels of lead in the '40s and '50s really were more likely to become violent criminals in the '60s, '70s, and '80s. \cite{nevin2000lead} compares changes in children’s blood lead levels in the United States with subsequent changes in IQ, based on norm comparisons for the Cognitive Abilities Test (CogAT) given to representative national samples of children in 1984 and 1992. The CogAT norm comparisons indicate shifts in IQ levels consistent with the blood lead to IQ relationship reported by an earlier study and population shifts in average blood lead for children under age 6 between 1976 and 1991. Furthermore, long-term trends in population exposure to gasoline lead were found to be remarkably consistent with subsequent changes in violent crime and unwed pregnancy. Long-term trends in paint and gasoline lead exposure are also strongly associated with subsequent trends in murder rates going back to 1900. The findings on violent crime and unwed pregnancy are consistent with published data describing the relationship between IQ and social behavior. The findings with respect to violent crime are also consistent with studies indicating that children with higher bone lead tend to display more aggressive and delinquent behavior. This analysis demonstrates that widespread exposure to lead is likely to have profound implications for a wide array of socially undesirable outcomes. Connecting deliquency to lead poisoning through indivudal studies helps to support the work of Nevins, that goes bigger picture in terms of leaded gasoline's effects. 

\medskip

\noindent \bibentry{nevin2007understanding}

\medskip

\cite{nevin2007understanding} continue with his theory of linkage between Lead poisoning and crime. However, this study from 2007 expands through the United States and into Britain, Canada, France, Australia, Finland, Italiy, West Germany, and New Zealand, with consistent confirmation with the linkage between Lead poisoning and crime, specifically index crime, burglary, and violent crime. The impact of blood lead is also evident in age-specific arrest and incarceration trends. Murder rates across USA cities also suggested that murder could be especially associated with more severe cases of childhood lead poisoning.

\medskip

\noindent \bibentry{wolpaw2007environmental}

\medskip

\cite{wolpaw2007environmental} notes that exposure to lead in childhood can lead to psychological traits strongly associated with aggressive and criminal behavior. In the late 1970s in the United States, lead was removed from gasoline under the Clean Air Act. She uses the state-specific reductions in lead exposure that resulted from this removal to identify the effect of childhood lead exposure on crime rates. Mixed evidence supports an effect of lead exposure on murder rates, and little evidence indicates an effect of lead on property crime. She finds that the reduction in childhood lead exposure in the late 1970s and early 1980s was responsible for significant declines in violent crime in the 1990s and may cause further declines in the future. Moreover, she argues that the social value of the reductions in violent crime far exceeds the cost of the removal of lead from gasoline. The graphs of Reye’s data help to illustrate her argument through showing consistent patterns of crime in different states and by removing the 22-year lag in between lead exposure and crime commitment. 

\medskip

\noindent \bibentry{chiodo07}

\medskip

The results of this 2007 study show a relation between blood lead level and neurobehavioral outcome in 7-year-old children (with 506 subjects  in the study). This specific age of investigation complicates the assumption of a 22-year lag found by Nevins and the 23-year found by Reyes in terms of criminality. Higher lead levels were associated significantly with decreased scores on measures of intelligence (i.e., overall, performance and verbal IQ), lengthened reaction time, hyperactivity, and social and delinquent behavior problems. While other studies touch upon this, the hyperactivity and lengthened reaction time are further investigated than in others. This article brings in the question of what a society becomes when poisoned by lead. This clearly instigates lower performance in schools in the area. In addition, I question the impact of slower reaction times. I question how this comes into play in terms of criminality and getting caught, as well as how this simply affects traffic safety in the area. This study also enforced that because of neurobehavioral outcomes assessed, there is no threshold above zero lead levels that appear to be ``safe''.

\medskip

\noindent \bibentry{Hou2013}

\medskip

In this study, Hou et al. investigated the relationship between lead poisoning and the intellectual and neurobehavioral capabilities of children. The background characteristics of the research subjects were collected by questionnaire survey. Intelligence was assessed using the Gesell Developmental Scale. The Achenbach Child Behavior Checklist was used to evaluate each child’s behavior. In result, blood lead levels were significantly negatively correlated with the developmental quotients of adaptive behavior, gross motor performance, fine motor performance, language development, and individual social behavior (P < 0.01). Compared with healthy children, more children with lead poisoning had abnormal behaviors, especially social withdrawal, depression, and atypical body movements, aggressions and destruction. This is consistent with other studies' findings for adverse effects on intellectual development in children.

\medskip

\noindent \bibentry{olympio2010surface}

\medskip

This study from 2010 complicates the connection between criminality and lead poisoning by focusing on antisocial behavior in connection to lead in Brazil. The focus on lead’s causing of antisocial behavior is likely connected to committing crimes. In the study, the connection is referred to as ``antisocial/delinquency'' as if they are one in the same. However, there is not a direct causation. This highlights antisocial behavior as another public health issue with lead, along with IQ level, math and reading skills, attention disorders, and criminality. 

\medskip

\noindent \bibentry{Feigenbaum2016}

\medskip

This paper studies the effect of cities' use of lead pipes on homicide between 1921 and 1936. Lead water pipes exposed entire city populations to much higher doses of lead than have previously been studied in relation to crime. The estimates suggest that cities’ use of lead service pipes considerably increased city-level homicide rates. Not only does this enforce the connection of Lead poisoning and crime, but it also explains the background of how Lead poisoning is able to have such a presence in our history. Acknowledging pipes as a major factor in this connection helps to get a clear picture of this phenomenon. 


%\section{Works Cited}

\bibliographystyle{plainnat}
\bibliography{Pb_LiteratureCOMPLETE}
%\bibliography

\end{document}

